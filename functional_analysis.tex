\documentclass[a4paper, 10pt]{article}

\usepackage[margin=1in]{geometry}
\usepackage{amsfonts, amsmath, amssymb, amsthm}
\usepackage[utf8]{inputenc}
\usepackage[english, main=ukrainian]{babel}
\usepackage{pgfplots}
\usepackage{bm}
\usepackage{physics}
\usepackage[unicode]{hyperref}
\usepackage{tikz-cd}
\usepackage{enumitem}
\usepackage{graphicx}
\usepackage{pgfplots}
\usepackage{pdfpages}
\usepackage{caption}
\usepackage{float}

\usepgfplotslibrary{fillbetween}

\usetikzlibrary{spy}
\usetikzlibrary{fit,matrix}

\def\rightproof{$\boxed{\Rightarrow}$ }

\def\leftproof{$\boxed{\Leftarrow}$ }

\newtheoremstyle{theoremdd}
  {\topsep}
  {\topsep}
  {\normalfont}
  {0pt}
  {\bfseries}
  {}
  { }
  {\thmname{#1}\thmnumber{ #2}\textnormal{\thmnote{ \textbf{#3}\\}}}

\theoremstyle{theoremdd}
\newtheorem{theorem}{Theorem}[subsection]
\newtheorem{definition}[theorem]{Definition}
\newtheorem{example}[theorem]{Example}
\newtheorem{proposition}[theorem]{Proposition}
\newtheorem{remark}[theorem]{Remark}
\newtheorem{lemma}[theorem]{Lemma}
\newtheorem{corollary}[theorem]{Corollary}

\newcommand\thref[1]{\textbf{Th.~\ref{#1}}}
\newcommand\defref[1]{\textbf{Def.~\ref{#1}}}
\newcommand\exref[1]{\textbf{Ex.~\ref{#1}}}
\newcommand\prpref[1]{\textbf{Prp.~\ref{#1}}}
\newcommand\rmref[1]{\textbf{Rm.~\ref{#1}}}
\newcommand\lmref[1]{\textbf{Lm.~\ref{#1}}}
\newcommand\crlref[1]{\textbf{Crl.~\ref{#1}}}

\renewcommand{\qedsymbol}{$\blacksquare$}


\makeatletter
\renewenvironment{proof}[1][Proof.\\]{\par
\pushQED{\hfill \qed}%
\normalfont \topsep6\p@\@plus6\p@\relax
\trivlist
\item\relax
{\bfseries
#1\@addpunct{.}}\hspace\labelsep\ignorespaces
}{%
\popQED\endtrivlist\@endpefalse
}
\makeatother

\title{Функціональний аналіз \\ І курс магістратура, 2 семестр}
    	
\begin{document}
\maketitle
\newpage
%\tableofcontents
%\newpage
\subsection{Деякі вступні слова}
Деякі означення зі загальної топології, метричних просторів, лінійної алгебри та теорії міри вважатимуться відомими.

\begin{definition}
Задано $E$ -- векторний простір над полем $k$ (у рамках даного курсу переважно будуть поля $\mathbb{R}, \mathbb{C}$).\\
Векторний простір $E$ буде \textbf{топологічним}, якщо
\begin{align*}
\text{1) задана стандартна топологія на полі } k \\
\text{2) операції $+ \colon E \times E \to E$ (додавання) та $\cdot \colon k \times E \to E$ (множення на скаляр) -- неперервні }
\end{align*}
\end{definition}
\noindent
Мабуть, варто розписати детально, що ми матимемо в такому разі. Тимчасово позначу додавання за відображення $\text{add} \colon E \times E \to E$ та множення на скаляр за відображення $\text{scalar} \colon k \times E \to E$.\\
Оберемо будь-яку точку $(x,y) \in E \times E$. Тоді на ній $\text{add}$ неперервне, тобто $\forall U$ -- відкритий окіл $\text{add}(x,y): \exists V$ -- відкритий окіл точки $(x,y): \text{add}(V) \subset U$. Зауважимо, що для $V$ -- відкритого окола $(x,y)$ -- існують відкриті околи $V_x,V_y$, для яких $V_x \times V_y \subset V$. Далі, в нашому випадку $\text{add}(U) = \{ \text{add}(x,y) \mid (x,y) \in V \} = \{x+y \mid (x,y) \in V\} \supset \{x'+y' \mid x' \in V_x, y' \in V_y\} \overset{\text{позн.}}{=} V_x + V_y$.
\bigskip \\
Таким чином, $\forall U_{x+y}$ -- відкритий окіл $x+y: \exists V_x,V_y$ -- відкриті околи $x,y: V_x + V_y \subset U_{(x,y)}$.\\
Аналогічно $\forall U_{\lambda x}$ -- відкритий окіл $\lambda x: \exists V_\lambda, V_x$ -- відкриті околи $\lambda, x: V_\lambda \times V_x \subset U_{\lambda x}$.

\begin{remark}
Для топологічного векторного простору достатньо визначити окіл точки $0$.\\
Дійсно, всі інші околи $U_x \cong U_0$. В одну сторону в нас неперервне відображення $y \mapsto y+x$, а в іншу сторону -- теж неперервне $y \mapsto y - x$.
\end{remark}

\subsection{Лінійні нормовані простори}
\begin{definition}
Задано $E$ -- векторний простір над полем $k = \mathbb{R}$ або $k = \mathbb{C}$.\\
\textbf{Лінійним нормованим простором} називають векторний простір $E$ над полем $k$, на якій задається \textbf{норма} $\| \cdot \| \colon E \to k$, що задовольняє таким властивостям:
\begin{align*}
1) \forall x \in E: \| x \geq 0\|, \text{при цьому } \|x\| = 0 \iff x = 0 \\
2) \forall x \in E: \forall \lambda \in k: \| \lambda x\| = |\lambda| \|x\| \\
3) \forall x,y \in E: \|x+y\| \leq \|x\| + \|y\|
\end{align*}
\end{definition}

\begin{proposition}
Якщо $E$ -- лінійний нормвований простір, то $(E, \rho )$, де $\rho(x,y) = \|x-y\|$, автоматично утворює метричний простір.
\end{proposition}

\begin{proof}
1) $\rho(x,y) = \|x - y \| \geq 0$ -- перша властивість норми; $\rho(x,y) = \|x- y\| = 0 \iff x-y = 0 \iff x = y$;\\
2) $\rho(y,x) = \|y-x\| = \| (-1)(x-y) \| = |(-1)| \|x-y\| = \rho(x,y)$;\\
3) $\rho(x,z) = \| x - z\| = \|(x - y) + (y - z)\| \leq \|x-y\| + \|y-z\| = \rho(x,y) + \rho(y,z)$.
\end{proof}

\begin{corollary}
Якщо $E$ -- лінійний нормований простір, то $E$ -- автоматично лінійно топологічний простір.\\
\textit{Задамо просто окіл нуля як $B_0(r) = \{ x \in E \mid \|x\| < r\}$.}
\end{corollary}

\begin{proposition}[Властивості норми]
Задано $E$ -- лінійний норований простір. Тоді справедливо наступне:
\begin{enumerate}[nosep,wide=0pt,label={\arabic*)}]
\item $\| x -y \| \geq \left| \|x\| - \|y\| \right|$;
\item Нехай задана послідовність $\{x_n, n \geq 1\} \subset E: x_n \to x$. Тоді $\| x_n \| \to \|x\|$ при $n \to \infty$.
\end{enumerate}

\begin{proof}
Дещо я залишу без доведення:
\begin{enumerate}[label={\arabic*)},wide=0pt]
\item \textit{Вказівка: $\|x\| = \|x-y+y\|$ та $\|y\| = \|y-x+x\|$.}
\item $x_n \to x$, тобто це означає $\|x_n - x\| \to 0$. Отже, завдяки властивості 1), отримаємо $0 \leq \left| \| x_n \| - \| x \| \right| \leq \|x_n-x\| \to 0$ при $n \to \infty$. Таким чином, $\| x_n \| \to \|x\|$.
\end{enumerate}
Всі властивості доведені.
\end{proof}
\end{proposition}

\begin{theorem}
Задано $E$ -- лінійний нормований простір. Тоді існує лінійний нормований простір $\tilde{E}$, такий, що:
\begin{enumerate}[nosep,wide=0pt,label={\arabic*)}]
\item $E \subset \tilde{E}$ -- щільна;
\item $(E, \| \cdot \|_E)$ та $(\tilde{E}, \| \cdot \|_{\tilde{E}})$ -- ізометричні;
\end{enumerate}
\textit{Поки без доведення.}
\end{theorem}

\begin{example}
Розглянемо кілька прикладів лінійних нормованих просторів:
\begin{enumerate}[nosep, wide=0pt, label={\arabic*)}]
\item $\mathbb{C}^n$, \qquad $\|x\| = \sqrt{x_1^2 + \dots + x_n^2}$.
\item $C([0,1])$, \qquad $\|f\| = \displaystyle\max_{t \in [0,1]} |f(t)|$.
\item $X$ -- довільна множина, $M(X) = \{f \colon X \to \mathbb{C} \mid f \text{ -- обмежені на } X \}$, \qquad $\|f\| = \displaystyle\sup_{t \in X} |f(t)|$.
\end{enumerate}
\end{example}

\subsection{Гільбертові простори}
\begin{definition}
Задано $E$ -- векторний простір над полем $\mathbb{C}$.\\
Простір називається \textbf{передгільбертовим}, якщо задано відображення $(\cdot,\cdot) \colon E \times E \to k$, що наизвається \textbf{скалярним добутком}, що задовольняє таким умовам:
\begin{align*}
1)\forall x \in E: (x,x) \geq 0, \text{ при цьому } (x,x) = 0 \iff x = 0 \\
2)\forall x,y,z \in E, \forall \lambda,\mu \in L: (\lambda x + \mu y, z) = \lambda(x,z) + \mu(y,z) \\
3) \forall x,y \in E: (x,y) = \overline{(y,x)}
\end{align*}
Цей же скалярний добуток можна визначати над полем $\mathbb{R}$.
\end{definition}
\end{document}