\documentclass[a4paper, 10pt]{article}

\usepackage[margin=1in]{geometry}
\usepackage{amsfonts, amsmath, amssymb, amsthm}
\usepackage[utf8]{inputenc}
\usepackage[english, main=ukrainian]{babel}
\usepackage{pgfplots}
\usepackage{bm}
\usepackage{physics}
\usepackage[unicode]{hyperref}
\usepackage{tikz-cd}
\usepackage{enumitem}
\usepackage{graphicx}
\usepackage{pgfplots}
\usepackage{pdfpages}
\usepackage{caption}
\usepackage{float}

\usepgfplotslibrary{fillbetween}

\usetikzlibrary{spy}
\usetikzlibrary{fit,matrix}

\def\rightproof{$\boxed{\Rightarrow}$ }

\def\leftproof{$\boxed{\Leftarrow}$ }

\newtheoremstyle{theoremdd}
  {\topsep}
  {\topsep}
  {\normalfont}
  {0pt}
  {\bfseries}
  {}
  { }
  {\thmname{#1}\thmnumber{ #2}\textnormal{\thmnote{ \textbf{#3}\\}}}

\theoremstyle{theoremdd}
\newtheorem{theorem}{Theorem}[subsection]
\newtheorem{definition}[theorem]{Definition}
\newtheorem{example}[theorem]{Example}
\newtheorem{proposition}[theorem]{Proposition}
\newtheorem{remark}[theorem]{Remark}
\newtheorem{lemma}[theorem]{Lemma}
\newtheorem{corollary}[theorem]{Corollary}

\newcommand\thref[1]{\textbf{Th.~\ref{#1}}}
\newcommand\defref[1]{\textbf{Def.~\ref{#1}}}
\newcommand\exref[1]{\textbf{Ex.~\ref{#1}}}
\newcommand\prpref[1]{\textbf{Prp.~\ref{#1}}}
\newcommand\rmref[1]{\textbf{Rm.~\ref{#1}}}
\newcommand\lmref[1]{\textbf{Lm.~\ref{#1}}}
\newcommand\crlref[1]{\textbf{Crl.~\ref{#1}}}

\renewcommand{\qedsymbol}{$\blacksquare$}


\makeatletter
\renewenvironment{proof}[1][Proof.\\]{\par
\pushQED{\hfill \qed}%
\normalfont \topsep6\p@\@plus6\p@\relax
\trivlist
\item\relax
{\bfseries
#1\@addpunct{.}}\hspace\labelsep\ignorespaces
}{%
\popQED\endtrivlist\@endpefalse
}
\makeatother

\title{Функціональний аналіз \\ І курс магістратура, 2 семестр}
    	
\begin{document}
\maketitle
\newpage
%\tableofcontents
%\newpage
\subsection{Деякі вступні слова}
Деякі означення зі загальної топології, метричних просторів, лінійної алгебри та теорії міри вважатимуться відомими.

\begin{definition}
Задано $E$ -- векторний простір над полем $k$ (у рамках даного курсу переважно будуть поля $\mathbb{R}, \mathbb{C}$).\\
Векторний простір $E$ буде \textbf{топологічним}, якщо
\begin{align*}
\text{1) задана стандартна топологія на полі } k \\
\text{2) операції $+ \colon E \times E \to E$ (додавання) та $\cdot \colon k \times E \to E$ (множення на скаляр) -- неперервні }
\end{align*}
\end{definition}
\noindent
Мабуть, варто розписати детально, що ми матимемо в такому разі. Тимчасово позначу додавання за відображення $\text{add} \colon E \times E \to E$ та множення на скаляр за відображення $\text{scalar} \colon k \times E \to E$.\\
Оберемо будь-яку точку $(x,y) \in E \times E$. Тоді на ній $\text{add}$ неперервне, тобто $\forall U$ -- відкритий окіл $\text{add}(x,y): \exists V$ -- відкритий окіл точки $(x,y): \text{add}(V) \subset U$. Зауважимо, що для $V$ -- відкритого окола $(x,y)$ -- існують відкриті околи $V_x,V_y$, для яких $V_x \times V_y \subset V$. Далі, в нашому випадку $\text{add}(U) = \{ \text{add}(x,y) \mid (x,y) \in V \} = \{x+y \mid (x,y) \in V\} \supset \{x'+y' \mid x' \in V_x, y' \in V_y\} \overset{\text{позн.}}{=} V_x + V_y$.
\bigskip \\
Таким чином, $\forall U_{x+y}$ -- відкритий окіл $x+y: \exists V_x,V_y$ -- відкриті околи $x,y: V_x + V_y \subset U_{(x,y)}$.\\
Аналогічно $\forall U_{\lambda x}$ -- відкритий окіл $\lambda x: \exists V_\lambda, V_x$ -- відкриті околи $\lambda, x: V_\lambda \times V_x \subset U_{\lambda x}$.

\begin{remark}
Для топологічного векторного простору достатньо визначити окіл точки $0$.\\
Дійсно, всі інші околи $U_x \cong U_0$. В одну сторону в нас неперервне відображення $y \mapsto y+x$, а в іншу сторону -- теж неперервне $y \mapsto y - x$.
\end{remark}
\end{document}